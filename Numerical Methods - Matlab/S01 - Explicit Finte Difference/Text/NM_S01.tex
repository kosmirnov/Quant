\documentclass[a4paper,12pt]{article}
\usepackage{t1enc}
\usepackage[space]{grffile}
\usepackage[round, longnamesfirst]{natbib}  % Bindet den natbib-standard fuer das Zitieren ein
\usepackage{epsfig}
\usepackage{chngcntr}
\usepackage[latin1]{inputenc}   % Ermoeglicht Sonderzeichen direkt einzugeben
\usepackage[T1]{fontenc}        % Garantiert saubere Worttrennung bei Umlauten etc.
\usepackage{color}              % Farbpaket
\usepackage{amsmath,amsfonts,amssymb}   % ermoeglicht mathematische Sonderzeichen
\usepackage[english]{babel}     %
\usepackage{ae}                 %
%\usepackage{graphicx,subfigure}           % Ermoeglicht das Einbinden von Bildern in allen Formaten
\usepackage{graphicx}
\usepackage{longtable}          % zum erstellen von Tabellen ber mehrere Seiten
\usepackage{multirow}           % zum Verbinden von Zeilen innerhalb einer Tabelle
%\usepackage{pictexwd}           % PicTex, ein Graphikpaket
\usepackage{pst-all, multido}   % psTricks, ein Graphikpaket
\usepackage{url}
\usepackage{booktabs,caption,fixltx2e,array,dcolumn}
\usepackage{parskip}
\usepackage{etoolbox}
\usepackage{geometry}
\usepackage{tabularx}
\usepackage{caption}
\usepackage{subcaption}
%\usepackage{morefloats}
\usepackage[section]{placeins} % Mit FloatBarrier kann verhindert werden, dass Float in der nächsten Section angezeigt werden
\usepackage[bottom]{footmisc} % Fußnoten werden ganz unten angezeigt
\setcitestyle{authoryear, open={((},close={))}}

\newcolumntype{L}[1]{>{\raggedright\arraybackslash}p{#1}} % linksbündig mit Breitenangabe
\newcolumntype{C}[1]{>{\centering\arraybackslash}p{#1}} % zentriert mit Breitenangabe
\newcolumntype{R}[1]{>{\raggedleft\arraybackslash}p{#1}} % rechtsbündig mit Breitenangabe

\makeatletter
\interfootnotelinepenalty 10000
\patchcmd{\Ginclude@eps}{"#1"}{#1}{}{}
\makeatother

\makeatletter
\newcommand\tageq{%
  \ifmeasuring@\else
    \refstepcounter{equation}%
  \fi
  \tag{\theequation}%
}
\makeatother

\newcommand\fnote[1]{\captionsetup{font=tiny}\caption*{#1}}

%\parindent 0cm
%\renewcommand{\baselinestretch}{1}

\renewcommand{\bibnumfmt}[1]{#1.}
\bibpunct{(}{)}{;}{a}{,}{,}

\usepackage{setspace}
\makeatletter
\newcommand{\MSonehalfspacing}{%
  \setstretch{1.44}%  default
  \ifcase \@ptsize \relax % 10pt
    \setstretch {1.448}%
  \or % 11pt
    \setstretch {1.399}%
  \or % 12pt
    \setstretch {1.433}%
  \fi
}
\newcommand{\MSdoublespacing}{%
  \setstretch {1.92}%  default
  \ifcase \@ptsize \relax % 10pt
    \setstretch {1.936}%
  \or % 11pt
    \setstretch {1.866}%
  \or % 12pt
    \setstretch {1.902}%
  \fi
}
\makeatother
\MSonehalfspacing



%%%%%%%%%%%%%%%%%%%%%%%%%%%%%%%%%%%%%%%%%%%%%%%%%%%%%%%%%%%%%%%%%%%%%%%%%%%%%%%%%%%%%%%%%%%%%%%%%%%%%%%%%%%%%%%%%%%%%%%%%%%%%%%%%%%%


% ________________ EINRICHTEN DES DOKUMENTS ______________________%

%\bibliographystyle{plainnat}    % legt den Stil fuer das Inhaltsverzeichnis fest

\oddsidemargin 0.0in \evensidemargin 0.0in \textwidth 15.5cm \topmargin -0.8in \textheight 24.5cm
\parindent 0cm


\geometry{a4paper, top=25mm, bottom=2cm} 
\renewcommand{\baselinestretch}{1}


\setlength{\intextsep}{0.5cm} % Legt den Abstand zwischen Gleitobjekten und dem darüber und darunter angeordneten Fließtext fest.

%\oddsidemargin 0.1in \evensidemargin 0.1in \textwidth 15.5cm \topmargin -0.4in \textheight 24.5cm
%\parindent 0cm      % legt die Seitenraender fest

\pagestyle{plain}          % leere Kopfzeile, Seitennummer in der Mitte der Fusszeile

\newcommand{\bs}{\boldsymbol}  % shortcut zur Erzeugung von fetten Sympolen in der Mathe-Umgebung

%\renewcommand{\baselinestretch}{1.25}
% 1,5 -facher Zeilenabstand (Standard ist 1,2-facher Zeilenabstand, also 1,2*1,25 = 1,5

\begin{document}

\pagenumbering{roman}   % roemische Zahlen zur Seitennumerierung


%%%%%%%%%%%%%%%%%%%%%%%%%%%%%%%%%%%%%%%%%%%%%%%%%%%%%%%%%%%%%%%%%%%%%%%%%%%%%%%%%%%%%%%%%%%%%%%%%%%%%%%%%%%%%%%%%%%%%%%%%%%%%%%%%%%%%%%5


\begin{titlepage}       % Umgebung fuer Titelseite, frei gestaltbar

\thispagestyle{empty}   % keine Numerierung auf Titelseite

\begin{center}
\includegraphics[width=13cm]{Bild.png}
\end{center}

\begin{center}

\vspace*{1.5cm}
{\bf  \Large Assignment \#1:: The Explicit Finite Difference Method} \\
\vspace*{2cm} 
\textbf{B473 Numerical Methods in Finance}
\\
Prof. Dr.-Ing. Rainer Sch\"obel
\\
\vspace*{0.5cm} 
Winter Term 2016-2017\\
\end{center}

\vfill
\begin{flushright}
    \emph{Konstantin Smirnov}\\
    \textit{Student-ID: 3980253}\\
     \textit{ Economics and Finance}\\
 T\"ubingen, \today
  




\end{flushright}



% 
% \begin{center}
% $ $			% oeffnet und schliesst eine Matheumgebung (Trick, um den Titel nach unten zu rutschen
% \vspace{4cm}
% 
% {\LARGE TITEL}
% \vskip 4cm
% 
% Diese Seite ist frei gestaltbar
% \end{center}

\end{titlepage}

\newpage                % erzwingt an dieser Stelle einen Seitenumbruch

\listoffigures
\newpage


\tableofcontents

\newpage
\pagenumbering{arabic}      % Seitenzahlen wieder arabisch numerieren
\setcounter{page}{1}        % Ruecksetzen des Seitenzahlzaehlers auf 1




\section{Introduction}

\section{References}

Brennan, Michael, and Eduardo Schwartz. Finite Difference Methods and Jump Processes and the Pricing of Contingent Claims: A Synthesis. \textit{Journal of Financial and Quantitative Analysis 13}, 1978, 461-474.\\

Brandimarte, Paolo. \textit{Numerical Methods in Finance and Economics - A MATLAB-Based Introduction}, $2^{nd}$ ed., John Wiley \& Sons, Hoboken NJ, 2006.\\

Fusai, Gianluca, and Andrea Roncoroni. \textit{Implementing Models in Quantitative Finance: Methods and Cases.} Springer-Verlag, Berlin Heidelberg New York 2008.\\

Hirsa, Ali. \textit{Computational Methods in Finance}. Chapman \& Hall/ CRC Press, Boca Raton FL, 2013.
\end{document}